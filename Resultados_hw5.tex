
\documentclass[a4paper]{article}

%% Language and font encodings
\usepackage[spanish]{babel}
\usepackage[utf8]{inputenc}
\usepackage[T1]{fontenc}

%% Sets page size and margins
\usepackage[a4paper,top=3cm,bottom=2cm,left=3cm,right=3cm,marginparwidth=1.75cm]{geometry}

%% Useful packages
\usepackage{amsmath}
\usepackage{graphicx}
\usepackage[colorinlistoftodos]{todonotes}
\usepackage[colorlinks=true, allcolors=blue]{hyperref}

\title{Tarea 5: métodos computacionales}
\author{Robinson Orlando Villamil benavides}

\begin{document}
\maketitle



\section{Estimación baynesiana de parámetros}
Mediante estimación baynessiana de parámetros con Monte Carlo, se obtuvo la masa del disco estelar, el bulbo galáctico y un halo de materia oscura para la galaxia a partir de datos de velocidad radial contra radio.



\begin{figure}[!h]
%\centering
\includegraphics[width = 0.8\textwidth]{modelo.png}
\caption{\label{fig:01} Gráfica de los datos y el modelo a partir de estimación de parámetros baynesiana. }
\end{figure}
\newpage



\end{document}
